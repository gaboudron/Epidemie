\documentclass{article}
\usepackage[utf8]{inputenc}
\usepackage[T1]{fontenc}
\usepackage{amsmath,amssymb,latexsym,amsthm}
\usepackage[french]{babel}
\usepackage{listings}
\usepackage{hyperref}
\usepackage{enumitem}
\usepackage{graphicx}
\usepackage{setspace}
\usepackage{algorithm}
\usepackage{algorithmic}
\usepackage{amsmath}
\usepackage{tabto}

\newlength{\larg}
\setlength{\larg}{14.5cm}

\title{
{\rule{\larg}{1mm}}\vspace{5mm}
\begin{tabular}{p{0cm} r c}
   & {\huge \textbf{Projet Informatique (Simulation) TP3}} \\
   & \\
   &{\huge Projet Pandémie}
\end{tabular}\\
\vspace{2mm}
{\rule{\larg}{1mm}}
\vspace{2mm} \\
}
  
\author{\begin{tabular}{p{13.7cm}}
Bourdon Gabriel \hspace{5cm} BARTIN Nallé 
\end{tabular}\\
\\ }\medskip
\begin{document}
\maketitle
\newpage
\tableofcontents
\newpage
\section{Introduction}
Ce projet consistait à reproduire l'évolution de la propagation d'une épidémie au cours du temps de manière simple grâce à un générateur de nombre aléatoire. Nous avons utilisé pour cette simulation un monde d'une surface de $300 \times 300$ dans lequel nous avons crée $20000$ humains différents selon des paramètres de risques d'infections dont 20 déjà infectés au début.

\section{Méthodes}
Nous avons donc crée plusieurs $class$ dont: \begin{itemize} \item Monde; \item Humain; \end{itemize}

Ainsi que d'autres qui nous ont servit pour faire nos tests. Cependant pour faire une expérience comme celle-ci avec toutes ces $class$ il nous a fallut donné certains paramètres à nos humains, ainsi que certaines méthodes pour pouvoir gérer la modification des status de nos humains car ceux-ci peuvent avoir différents status $(Infected, Susceptible, Recovered, Exposed)$ à chaque pas de temps que nous avons représenté grâce à des boucles $for$.
Ces status sont pris par chaque humain de manière indépendante grâce à la fonction \textbf{negExp}, ainsi que les fonctions qui permettent le changement de status comme avec la fonction qui nous permet de savoir si un humain va être infecter ou non.

\section{Conclusion}
Pour finir nous avons utilisé le Mersenne Twister un simulateur de nombre aléatoire celui-ci doit être initialisé correctement pour donné des résultats coéhrents cela dépend de la graine entrée dans le générateur.
Cela nous permet ensuite de générer des résultats en grand nombre assez rapidement.


\end{document}